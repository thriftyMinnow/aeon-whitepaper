\section{AEON Differentiation}

With the astounding plethora of "alt coins" now available, it is worth noting how AEON differs from others, and specifically how it improves upon some of the similar offerings.\\
\\
The first thing to note is that within the alt coin universe, there are different classes of blockchain and coin.  There are "smart contract" blockchains (i.e. Ethereum) which provide a mechanism to manage complex transactions such as business contracts and decentralized application ("dApp") hosting.  There are also "token" coins (i.e. STEEM token) which supply a payment mechanism for use of a particular decentralized application or service.\\
\\ 
AEON is a "currency" coin, intended to provide an alternative to local fiat currencies.  Therefore, this section will provide comparisons only to other well-known currency coins.\\
\\
(NOTE: for more details on the concepts in this section, see section \ref{secImp} \textit{AEON Implementation}.)

\subsection{vs. Bitcoin}
\label{secVsBitcoin}
Bitcoin is the best-known of all crypto currency blockchains, as it was the first to achieve a measure of success. There are considerable differences between AEON and Bitcoin, in the areas of privacy and usability.

\subsubsection{Privacy and Transaction Linkages}
Regarding the critical feature of privacy, Bitcoin falls short of the AEON blockchain. In order to maintain privacy of individual expenditures, it must be exceedingly difficult for an outside party to link a transaction back to its owner.\\
\\
Consider that each transaction consists of some \textit{inputs} (coins which are being spent) and some \textit{outputs} (one or more addresses which receive the spent coins).  Additionally, each input in a transaction actually links to an output of a \textit{previous} transaction, forming a set of transaction paths.\\
\\
In Bitcoin, these transaction linkages are explicitly transparent on the blockchain. Any blockchain explorer can follow the graph, which has allowed for sophisticated analysis to de-anonymize transactions. This privacy issue is alleviated in various ways, such as creating a unique address for every transaction, using centralized "mixers" to randomly "mix up" several people's Bitcoins, and employing methods to hide IP addresses when making transactions.  The fact remains, however, that the Bitcoin inputs and outputs can be directly followed on the blockchain.\\
\\
AEON resolves this privacy concern by \textit{intentionally obscuring} transaction linkages on the blockchain. Every transaction has a default number of "decoy" input links (also known as \textbf{mixins}). Anyone making a transaction can request a higher number of decoy inputs, to increase anonymity.  As the blockchain grows over time, the increasing number of decoy input links will make the overall graph of transactions exceedingly difficult, if not impossible, to correctly decipher.

\subsubsection{Mining and Barriers to Participation}
Bitcoin uses a SHA-256 Proof-of-Work (PoW) algorithm which is dependent primarily on CPU power, and there are several specialized ASIC hardware devices made for mining Bitcoin. This has driven the hashrate high enough that currently only ASIC hardware mining is profitable.  The result is that the average person with a PC cannot readily participate in the transaction validation process of mining and acquiring Bitcoins.\\
\\
AEON uses a CPU-friendly PoW algorithm that limits the advantage of GPU’s and is ASIC resistant. This allows almost anyone with a PC to participate in mining and acquiring AEON. 

\subsubsection{General Usability and Transactions-Per-Second}
Regarding usability and the vision of a lightweight digital currency for everyone, AEON has distinct advantages over Bitcoin.\\
\\
The maximum blocksize of Bitcoin (1 MB) and the block creation time of 10 minutes limits the transactions-per-second (TPS) processing power to no more than 7 TPS.  This low TPS severely hinders the ability of the Bitcoin blockchain to process transactions for the masses.  While there are ways to remedy this limitation, the Bitcoin community has not achieved consensus on doing so.  Thus, the coin has become more of a high-end investment with limited use as a currency, not unlike physical gold coins versus U.S. dollar bills.\\
\\
AEON solves the TPS limitation by using an algorithm to automatically adjust the maximum block size up or down, based on the previous 100 blocks. This approach allows the AEON blockchain to self-adjust it's TPS throughput as transaction traffic increases and decreases over time.

\subsection{vs. Litecoin}
Litecoin was started as a fork of the Bitcoin code in 2011, with the goal of being a lighter-weight currency, offering low-cost transactions with fast confirmation status.

\subsubsection{Privacy and Transaction Linkages}
See the description of the privacy issues in section \ref{secVsBitcoin} \textit{vs. Bitcoin}. Litecoin has the same issues as Bitcoin.

\subsubsection{Mining and Barriers to Participation}
Litecoin uses a Proof-of-Work algorithm called \textbf{scrypt} which depends not only on the CPU, but also on fast access to a memory area.  This PoW makes it difficult to develop specialized ASIC hardware, and renders GPU's to be only about 10X faster than CPU's.  This is a great improvement over Bitcoin, but still leaves mining largely in the hands of those who can purchase high end graphics cards or special built ASICS.\\
\\
AEON actually uses an improved version of the scrypt algorithm which employs a larger memory area. The result is that AEON is even more resistant than Litecoin to specialized hardware, and the GPU cards do not have as great an advantage over the CPU.  This ensures that CPU mining with an average PC is an option for everyone.

\subsubsection{General Usability and Transactions-Per-Second}
Being a "lightweight Bitcoin" it is no surprise that Litecoin is able to boast roughly 8 times the TPS of Bitcoin.  This brings the maximum capacity of Litecoin up to 56 transactions per second. For comparison, credit card processors typically see tens of thousands of transactions per second. While the Litecoin network can currently process transactions fast enough for its volume of users, at some point -- long before Litecoin can become a currency for the masses -- its TPS must be greatly increased.\\
\\
See the prior section \ref{secVsBitcoin} \textit{vs. Bitcoin} for a description of AEON's solution to the TPS limitation.

\subsection{vs. Monero}
It is public knowledge that AEON is a fork of the Monero project, and it continues to incorporate improvements directly from the Monero code base.  In fact, the Development team for AEON consists largely of Monero developers who also work on AEON.  Since Monero itself is well-known as a security/privacy coin, it requires some attention, to address exactly why AEON might be preferred.

\subsubsection{General Usability and Mobile-Friendliness}
The advantages that AEON has over Monero are in the area of being lightweight and mobile-friendly.  AEON has chosen a different Proof-of-Work algorithm which requires half the CPU memory and allows for faster verification of the blockchain.  The blockchain is pruned in a manner that keeps it smaller than Monero's, resulting in faster blockchain synchronization.  AEON also allows the option for a limited number of fast, low-fee transfers (which are more traceable on the blockchain) for non-sensitive payments.  Monero, on the other hand, requires all payments to be fully anonymized which adds to the validation times and blockchain size.\\
\\
All of these aspects, among others, put AEON in a better position to be the secure, private currency that can be used by the general public with cell phones and tablets on the go.

\subsection{vs. Dash}
Dash stands for "digital cash" and is meant to work like physical cash when purchasing items online or in stores.  Like AEON, Dash embraces the importance of Security and Privacy.  There are some disadvantages, however, when comparing this coin to AEON.

\subsubsection{General Usability and Transactions-Per-Second}
In November, 2017, the Dash blockchain hard-forked to double it's maximum blocksize, to 2 MB. That change allowed Dash to process roughly 48 transactions per second.  For comparison, credit card processors typically see tens of thousands of transactions per second.  At some point the Dash TPS must be increased again, and likely again after that.  This continued increasing of TPS via disruptive blockchain modifications is not conducive to massive adoption.\\
\\
See section \ref{secVsBitcoin} \textit{vs. Bitcoin} for a description of AEON's solution to the TPS limitation.

\subsubsection{Mining, Governance and Barriers to Participation}
Dash uses a Proof-of-Work algorithm which is dependent primarily on CPU power, and the network welcomes the use of specialized hardware for mining. This has driven the hashrate high enough that currently only ASIC hardware mining is profitable.  The result is that the average person with a PC cannot readily participate in the transaction validation process of mining.\\
\\
Additionally, Dash implements a complex form of Governance consisting of a 2nd tier network node, called a Masternode. In order to own a Masternode, one must obtain and hold 1000 Dash.  (At the time of this writing, this is an investment of roughly \$200,000 USD.)  Only Masternode owners vote on proposed enhancements to the coin, as well as prioritize which projects get paid from the development fund.  Similar to AEON, new coins are disbursed as a block reward when a miner successfully validates a block of transactions.  But unlike AEON, the miner must split the block reward between the Masternodes and the Development Fund.\\
\\
Because special hardware is required to mine Dash, and a large monetary investment is necessary to participate in the governance of the currency, the barriers to participation are much higher than with AEON's simple open source project model.  Even to submit a proposal for a vote by the Masternode owners, costs a fee of 5 Dash (roughly \$1000 USD at the time of this writing).  These factors work against a true decentralization for a currency.\\
\\
AEON uses the traditional Open Source Model of participation and governance, which has been shown to work well for many large technology efforts for many years. It allows a diverse community to grow organically, which is an advantage for AEON's plans to become widely used by all walks of life.

\subsection{vs. Zcash}
Zcash is another fork of the Bitcoin code base, with the intent of adding the element of privacy to the blockchain.

\subsubsection{Privacy and Shielded Transactions}
Zcash achieves privacy by using a cryptographic approach called "zero-knowledge cryptography" to create "shielded transactions".  However, this is not the default option, and there is no limit to the number of non-private transactions in each block.  A recent report by ICO research firm Satis Group ("Cryptoasset Market Coverage Initiation: Valuation", August 30, 2018) states:\\
\\ 
"Only ~5\%  of the Zcash network uses 'shielded' addresses currently, with the rest of the addresses being used for transactions functionally and technically no different than Bitcoin."\\
\\
The paper concludes that since there are so many more addresses in the blockchain that are not private, the Zcash network as a whole is not fungible.  Meaning the coins in any given wallet could possibly be traced back to their prior transactions.  (This is important, because nobody wants to find out that the coins in their wallet were used previously to commit a crime, etc.)\\
\\
In contrast, the research states specifically that Monero -- and therefore we can conclude AEON as well -- is a fungible network.  Both Monero and AEON default to a private transaction, and AEON only allows at most 10\% of the transactions in any block to be switched to non-private.

\subsubsection{Mining and Barriers to Participation}
Zcash went away from the Bitcoin PoW algorithm, and implemented the Equihash PoW algorithm, to provide ASIC resistance, and allow people to mine with their GPU’s and CPU’s.  The problems with the Equihash algorithm are that 1) it does not limit the advantage of GPU over CPU, and 2) an ASIC was recently developed which could mine Zcash, threatening the future viability of GPU and CPU mining.   As of the date of this paper, the Zcash project has not created a fork of the blockchain to provide ASIC resistance once again.\\
\\
AEON's PoW, as stated previously, is more CPU friendly, and more ASIC resistant.  In addition, when an ASIC was successfully developed for the CryptoNight-Lite algorithm, the AEON developers quickly moved to fork the blockchain to regain ASIC resistance.

\subsubsection{General Usability and Transactions-Per-Second}
Zcash improved upon Bitcoin by doubling the maximum block size to 2 MB, and decreasing the block interval down to 2.5 minutes.  However the transaction size is larger in Zcash, so the maximum TPS will only reach about 26, and could be far less if more of the transactions happen to be shielded.  As with Bitcoin, the TPS throughput will need to be increased into the thousands before Zcash will become a currency used by the masses.\\
\\
See section \ref{secVsBitcoin} \textit{vs. Bitcoin} for a description of AEON’s solution to the TPS limitation.

\subsection{vs. Ripple}
Of all the currencies discussed in this section, Ripple is one which may not belong.  The Ripple blockchain is open source, but Ripple is also a private company.  From their homepage:  "Ripple connects banks, payment providers, digital asset exchanges and corporates via RippleNet to provide one frictionless experience to send money globally."  Consider the ways in which Ripple's vision is different from AEON:

\begin{itemize}
	\item Ripple does not decentralize the management of personal wealth; it seeks to strengthen the ability of central entities to control the movement of wealth on a global scale.
	\item Working with banks, Ripple does not provide privacy, but instead provides full traceability of funds.
	\item There is no mining process, by which individuals can receive coins for themselves.  All coins have been produced, and the Ripple company releases a certain number of coins per month.
\end{itemize}

